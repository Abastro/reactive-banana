% reactivebanana-Hr.tex
%
% Entry for the  Haskell Communities and Activities Report
%
% http://wiki.haskell.org/Haskell_Communities_and_Activities_Report
% 
\begin{hcarentry}[new]{reactive-banana}
\report{Heinrich Apfelmus}%05/11
\status{active development}
\makeheader

Reactive-banana is a small library for functional reactive programming (FRP).

The goal is to create a solid foundation for anything FRP-related.
\begin{itemize}
\item Users can finally start experimenting with graphical user interfaces based on FRP since the library can be hooked into \emph{any} existing event-based framework and comes with ample documentation.
\item FRP implementors will have a reference for a simple semantics with a working implementation.
\item No more spooky time leaks and efficiency concerns. Predicting space \& time usage should be straightforward.
\end{itemize}

Version 0.2 of the reactive-banana library has been released on Hackage. It provides a solid push-based implementation of a subset of the semantics for FRP pioneered by Conal Elliott.

Current development focuses on providing tutorials, documentation and examples for the library. Furthermore, the author is writing an example application to test and refine the FRP approach to GUI programming.

\FurtherReading
\begin{compactitem}
\item  Cabal package and link to source code:  \url{http://hackage.haskell.org/package/reactive}
\item  Developer blog:  \url{http://apfelmus.nfshost.com/blog.html}
\end{compactitem}
\end{hcarentry}
